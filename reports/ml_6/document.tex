\input{config/preamble}

\input{config/macros}

\begin{document}

\input{sections/title}

\section{Кластеризация}
\label{sec:purpose}



\subsection{Загрузите данные ex6data1.mat из файла.}

\begin{lstlisting}
        dataset = read_matlab_file(DATA_PATH_1)
        x = dataset.get("X")
\end{lstlisting}

\subsection{Реализуйте функцию случайной инициализации K центров кластеров.}

\begin{lstlisting}
def init_centers(x, centroids_count):
    randomly_ordered_indexes = np.random.permutation(x.shape[0])
    return x[randomly_ordered_indexes[0:centroids_count], :]
\end{lstlisting}

\subsection{Реализуйте функцию определения принадлежности к кластерам.}

\begin{lstlisting}
def k_mean_algorithm(x, centers, max_iterations):
    indices = np.zeros((x.shape[0], 1))
    centroids_history = np.zeros((max_iterations, centers.shape[0], centers.shape[1]))
    centroids = centers.copy()
    for iteration in range(0, max_iterations):
        indices = find_closest_center(x, centroids)
        centroids_history[iteration] = centroids
        centroids = reassign_centres(x, indices, centroids)
    return centroids, indices, centroids_history
\end{lstlisting}

\subsection{Реализуйте функцию пересчета центров кластеров.}

\begin{lstlisting}
def reassign_centres(x, current_indices, centers):
    centroids_amount = centers.shape[0]
    for centroid_index in range(0, centroids_amount):
        assigned_features = x[np.where(np.any(current_indices == centroid_index, axis=1)), :]
        if assigned_features.size > 0:
            centers[centroid_index, :] = np.mean(assigned_features, axis=1)
    return centers
\end{lstlisting}

\subsection{Реализуйте алгоритм K-средних.}

\begin{lstlisting}
def k_mean_algorithm(x, centers, max_iterations):
    indices = np.zeros((x.shape[0], 1))
    centroids_history = np.zeros((max_iterations, centers.shape[0], centers.shape[1]))
    centroids = centers.copy()
    for iteration in range(0, max_iterations):
        indices = find_closest_center(x, centroids)
        centroids_history[iteration] = centroids
        centroids = reassign_centres(x, indices, centroids)
    return centroids, indices, centroids_history
\end{lstlisting}

\subsection{Постройте график, на котором данные разделены на K=3 кластеров, а также траекторию движения центров кластеров в процессе работы алгоритма}

\begin{figure}[h]
\centering
    \includegraphics[totalheight=7cm]{1.png}
  \label{sec:purpose:payings}
\end{figure}

\begin{figure}[h]
\centering
    \includegraphics[totalheight=7cm]{2.png}
  \label{sec:purpose:payings}
\end{figure}

\subsection{Загрузите изображение.}

\begin{lstlisting}
dataset = read_matlab_file(DATA_PATH_2)
x = dataset.get("A")
\end{lstlisting}

\subsection{С помощью алгоритма K-средних используйте 16 цветов для кодирования пикселей.}

\begin{lstlisting}
def translate_mat_to_compressed_img(img_data, colors_amount, iterations_amount, img_path):
    data_shape = img_data.shape
    img_to_save = Image.fromarray(img_data.astype('uint8'), 'RGB')
    img_to_save.save(img_path)
    img_data = img_data.reshape((data_shape[0] * data_shape[1], data_shape[2]))
    img_data = img_data / 255

    initial_centroids = init_centers(img_data, colors_amount)
    centroids, indices, centroids_history = k_mean_algorithm(img_data, initial_centroids, iterations_amount)
    classified_image_data = centroids[indices, :] * 255
    classified_image_data = classified_image_data.reshape(data_shape)

    img_to_save = Image.fromarray(classified_image_data.astype('uint8'), 'RGB')
    img_to_save.save(img_path + "_2.jpg")
\end{lstlisting}

\begin{figure}[h]
\centering
    \includegraphics[totalheight=7cm]{bird.png}
  \label{sec:purpose:payings}
\end{figure}

\begin{figure}[h]
\centering
    \includegraphics[totalheight=7cm]{bird_2.png}
  \label{sec:purpose:payings}
\end{figure}

\subsection{Насколько уменьшился размер изображения? Как это сказалось на качестве?}

Размер после кластеризации 768 байт, размер по итогу уменьшился в 64 раза. Качество ухудшилось, стало меньше цветов.

\subsection{Реализуйте алгоритм K-средних на другом изображении.}

\begin{figure}[h]
\centering
    \includegraphics[totalheight=7cm]{wolf.png}
  \label{sec:purpose:payings}
\end{figure}

\begin{figure}[h]
\centering
    \includegraphics[totalheight=7cm]{wolf_2.png}
  \label{sec:purpose:payings}
\end{figure}

\subsection{Реализуйте алгоритм иерархической кластеризации на том же изображении. Сравните полученные результаты.}

\begin{figure}[h]
\centering
    \includegraphics[totalheight=7cm]{3.png}
  \label{sec:purpose:payings}
\end{figure}


\end{document}