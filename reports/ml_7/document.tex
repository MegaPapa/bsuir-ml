\input{config/preamble}

\input{config/macros}

\begin{document}

\input{sections/title}

\section{Метод главных компонент}
\label{sec:purpose}

Набор данных ex7data1.mat представляет собой файл формата *.mat (т.е. сохраненного из Matlab). Набор содержит две переменные X1 и X2 - координаты точек, для которых необходимо выделить главные компоненты.

Набор данных ex7faces.mat представляет собой файл формата *.mat (т.е. сохраненного из Matlab). Набор содержит 5000 изображений 32x32 в оттенках серого. Каждый пиксель представляет собой значение яркости (вещественное число). Каждое изображение сохранено в виде вектора из 1024 элементов. В результате загрузки набора данных должна быть получена матрица 5000x1024.


\subsection{Загрузите данные ex7data1.mat из файла.}

\begin{lstlisting}
        dataset = read_matlab_file(DATA_PATH_1)
        x = dataset.get("X")
\end{lstlisting}

\subsection{Постройте график загруженного набора данных.}

\begin{figure}[h]
\centering
    \includegraphics[totalheight=7cm]{1.png}
  \label{sec:purpose:payings}
\end{figure}

\subsection{Реализуйте функцию вычисления матрицы ковариации данных.}


\begin{lstlisting}
def pca(features):
    m = features.shape[0]
    covariance_matrix = (1 / m) * (features.T @ features)
    u, s, v = np.linalg.svd(covariance_matrix)
    return u, s

u, s = util.pca(x_scaled)
\end{lstlisting}

\subsection{Постройте на графике из пункта 2 собственные векторы матрицы ковариации.}

\begin{figure}[h]
\centering
    \includegraphics[totalheight=7cm]{2.png}
  \label{sec:purpose:payings}
\end{figure}

\subsection{Реализуйте функцию проекции из пространства большей размерности в пространство меньшей размерности с помощью метода главных компонент.}

\begin{lstlisting}
def project_data(x, u, e_vectors_amount):
    m = x.shape[0]
    projections = np.zeros((m, e_vectors_amount))
    for i in range(0, m):
        example = x[i, :].T
        projections[i, :] = example.T @ u[:, 0:e_vectors_amount]
    return projections
\end{lstlisting}

\subsection{Реализуйте функцию вычисления обратного преобразования.}

\begin{lstlisting}
def recover_data(projections, u, eigenvectors_amount):
    m = projections.shape[0]
    recovered_data = np.zeros((m, u.shape[0]))
    for i in range(0, m):
        projection = projections[i, :].T
        recovered_data[i, :] = projection.T @ u[:, 0:eigenvectors_amount].T
    return recovered_data
\end{lstlisting}

\subsection{Постройте график исходных точек и их проекций на пространство меньшей размерности (с линиями проекций).}

\begin{figure}[h]
\centering
    \includegraphics[totalheight=7cm]{3.png}
  \label{sec:purpose:payings}
\end{figure}

\subsection{Визуализируйте 100 случайных изображений из набора данных.}

\begin{figure}[h]
\centering
    \includegraphics[totalheight=7cm]{4.png}
  \label{sec:purpose:payings}
\end{figure}

\subsection{С помощью метода главных компонент вычислите собственные векторы.}

\begin{lstlisting}
        u, s = util.pca(x_scaled)
        projected_data = util.project_data(x_scaled, u, 36)
        recovered_features = util.recover_data(projected_data, u, 36)
\end{lstlisting}

\subsection{Визуализируйте 36 главных компонент с наибольшей дисперсией.}

\begin{figure}[h]
\centering
    \includegraphics[totalheight=7cm]{5.png}
  \label{sec:purpose:payings}
\end{figure}

\subsection{Как изменилось качество выбранных изображений?}

Ухудшилось.

\subsection{Визуализируйте 100 главных компонент с наибольшей дисперсией.}

\begin{figure}[h]
\centering
    \includegraphics[totalheight=7cm]{6.png}
  \label{sec:purpose:payings}
\end{figure}

\subsection{Как изменилось качество выбранных изображений?}

Ухудшилось относительно исходного.

\subsection{Используйте изображение, сжатое в лабораторной работе №6 (Кластеризация). С помощью метода главных компонент визуализируйте данное изображение в 3D и 2D.}

\begin{figure}[h]
\centering
    \includegraphics[totalheight=7cm]{7.png}
  \label{sec:purpose:payings}
\end{figure}

\begin{figure}[h]
\centering
    \includegraphics[totalheight=7cm]{8.png}
  \label{sec:purpose:payings}
\end{figure}

\subsection{Соответствует ли 2D изображение какой-либо из проекций в 3D?}

Да.


\end{document}