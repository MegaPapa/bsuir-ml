\input{config/preamble}

\input{config/macros}

\begin{document}

\input{sections/title}

\section{Метод опорных векторов}
\label{sec:purpose}

Набор данных ex5data1.mat представляет собой файл формата *.mat (т.е. сохраненного из Matlab). Набор содержит три переменные X1 и X2 (независимые переменные) и y (метка класса). Данные являются линейно разделимыми.

Набор данных ex5data2.mat представляет собой файл формата *.mat (т.е. сохраненного из Matlab). Набор содержит три переменные X1 и X2 (независимые переменные) и y (метка класса). Данные являются нелинейно разделимыми.

Набор данных ex5data3.mat представляет собой файл формата *.mat (т.е. сохраненного из Matlab). Набор содержит три переменные X1 и X2 (независимые переменные) и y (метка класса). Данные разделены на две выборки: обучающая выборка (X, y), по которой определяются параметры модели; валидационная выборка (Xval, yval), на которой настраивается коэффициент регуляризации и параметры Гауссового ядра.

Набор данных spamTrain.mat представляет собой файл формата *.mat (т.е. сохраненного из Matlab). Набор содержит две переменные X - вектор, кодирующий отсутствие (0) или присутствие (1) слова из словаря vocab.txt в письме, и y - метка класса: 0 - не спам, 1 - спам. Набор используется для обучения классификатора.

Набор данных spamTest.mat представляет собой файл формата *.mat (т.е. сохраненного из Matlab). Набор содержит две переменные Xtest - вектор, кодирующий отсутствие (0) или присутствие (1) слова из словаря vocab.txt в письме, и ytest - метка класса: 0 - не спам, 1 - спам. Набор используется для проверки качества классификатора.


\subsection{Загрузите данные ex5data1.mat из файла.}

\begin{lstlisting}
        dataset = read_matlab_file(DATA_PATH_1)
        x = dataset.get("X")
        y = dataset.get("y")
\end{lstlisting}

\subsection{Постройте график для загруженного набора данных.}

\begin{figure}[h]
\centering
	\includegraphics[totalheight=7cm]{1.png}
	\label{sec:purpose:payings}
\end{figure}

\subsection{Обучите классификатор с помощью библиотечной реализации SVM с линейным ядром на данном наборе.}

\begin{lstlisting}
def linear_svc(x, y, c):
    model = SVC(c, kernel="linear").fit(x, y)
    return model

\end{lstlisting}


\subsection{Постройте разделяющую прямую для классификаторов с различными параметрами C = 1, C = 100}

\begin{lstlisting}
optimazed_1 = linear_svc(x, y.ravel(), 1)
optimazed_100 = linear_svc(x, y.ravel(), 100)
\end{lstlisting}

\begin{figure}[h]
\centering
    \includegraphics[totalheight=7cm]{2.png}
    \label{sec:purpose:payings}
\end{figure}

\begin{figure}[h]
\centering
    \includegraphics[totalheight=7cm]{3.png}
    \label{sec:purpose:payings}
\end{figure}

Как видно из графиков, манипулирование С делает модель более чувствительной.

\subsection{Реализуйте функцию вычисления Гауссового ядра для алгоритма SVM.}

\begin{lstlisting}
def gauss_kernel(x, gamma=0.5):
    x0, x1 = x[:, 0], x[:, 1]
    return np.exp(-gamma * max(x0 - x1) ** 2)
\end{lstlisting}

\subsection{Загрузите данные ex5data2.mat из файла.}

\begin{lstlisting}
        dataset = read_matlab_file(DATA_PATH_2)
        x = dataset.get("X")
        y = dataset.get("y")
\end{lstlisting}

\subsection{Обработайте данные с помощью функции Гауссового ядра.}

\begin{lstlisting}
g_kernel = gauss_kernel(x)

lab5                 - [INFO ] -- Gauss kernell for x's in ex5data2 = 0.8443840167535899
\end{lstlisting}

\subsection{Обучите классификатор SVM.}

\begin{lstlisting}
optimazed_1 = rbf_svc(x, y.ravel(), 1, GAMMA)
optimazed_100 = rbf_svc(x, y.ravel(), 100, GAMMA)
\end{lstlisting}

\subsection{Визуализируйте данные вместе с разделяющей кривой.}

\begin{figure}[h]
\centering
    \includegraphics[totalheight=7cm]{4.png}
    \label{sec:purpose:payings}
\end{figure}

\begin{figure}[h]
\centering
    \includegraphics[totalheight=7cm]{5.png}
    \label{sec:purpose:payings}
\end{figure}

\subsection{Загрузите данные ex5data3.mat из файла.}

\begin{lstlisting}
        dataset = read_matlab_file(DATA_PATH_3)
        x = dataset.get("X")
        y = dataset.get("y")
        x_val = dataset.get("Xval")
        y_val = dataset.get("yval")
\end{lstlisting}

\subsection{Вычислите параметры классификатора SVM на обучающей выборке, а также подберите параметры C и Omega}

\begin{lstlisting}
optimal_c, optimal_gamma = find_best_params(x, y, x_val, x_val)

lab5                 - [INFO ] -- Optimal C = 0.5 ; Optimal gamma = 0.5
\end{lstlisting}

\subsection{Визуализируйте данные вместе с разделяющей кривой.}

\begin{figure}[h]
\centering
    \includegraphics[totalheight=7cm]{6.png}
    \label{sec:purpose:payings}
\end{figure}

\subsection{Загрузите данные spamTrain.mat из файла.}

\begin{lstlisting}
        dataset = read_matlab_file(SPAM_TRAIN_DATA)
        x = dataset.get("X")
        y = dataset.get("y")
\end{lstlisting}

\subsection{Обучите классификатор SVM.}

\begin{lstlisting}
optimazed = rbf_svc(x, y.ravel(), 1, optimal_gamma)
\end{lstlisting}

\subsection{Загрузите данные spamTest.mat из файла.}

\begin{lstlisting}
        dataset = read_matlab_file(SPAM_TEST_DATA)
        x_test = dataset.get("Xtest")
        y_test = dataset.get("ytest")
\end{lstlisting}

\subsection{Подберите параметры С и Omega}

\begin{lstlisting}
optimal_c = find_best_c(x, y, x_test, y_test)
optimized = rbf_svc(x, y, optimal_c)
\end{lstlisting}


\subsection{Реализуйте функцию предобработки текста письма}

\begin{lstlisting}
def process_text(content, vocabulary):
    content = content.lower()
    content = re.compile('<[^<>]+>').sub(' ', content)
    content = re.compile('[0-9]+').sub(' number ', content)
    content = re.compile('(http|https)://[^\s]*').sub(' httpaddr ', content)
    content = re.compile('[^\s]+@[^\s]+').sub(' emailaddr ', content)
    content = re.compile('[$]+').sub(' dollar ', content)
    content = re.split('[ @$/#.-:&*+=\[\]?!(){},''">_<;%\n\r]', content)
    content = [word for word in content if len(word) > 0]

    # Stem the email contents word by word
    stemmer = PorterStemmer()
    processed_content = []
    word_indices = []
    for word in content:
        word = re.compile('[^a-zA-Z0-9]').sub('', word).strip()
        word = stemmer.stem(word)

        if len(word) < 1:
            continue

        processed_content.append(word)

        try:
            word_indices.append(vocabulary.index(word))
        except ValueError:
            pass

    return word_indices
\end{lstlisting}

\subsection{Загрузите коды слов из словаря vocab.txt. Реализуйте функцию преобразования текста письма в вектор признаков}

\begin{lstlisting}
def read_vocabulary(file_path):
    vocabulary = np.genfromtxt(join('Data', file_path), dtype=object)
    return list(vocabulary[:, 1].astype(str)), vocabulary[:, 1].shape[0]
\end{lstlisting}    

\subsection{Проверьте работу классификатора на письмах из файлов emailSample1.txt, emailSample2.txt, spamSample1.txt и spamSample2.txt.}

\begin{lstlisting}
        is_spam('Email sample 1', PATH_TO_EMAIL_1_FILE, optimized, voc, voc_size)
        is_spam('Email sample 2', PATH_TO_EMAIL_2_FILE, optimized, voc, voc_size)
        is_spam('Spam sample 1', PATH_TO_SPAM_1_FILE, optimized, voc, voc_size)
        is_spam('Spam sample 2', PATH_TO_SPAM_2_FILE, optimized, voc, voc_size)
\end{lstlisting}

\subsection{Также можете проверить его работу на собственных примерах.}

\begin{lstlisting}
        is_spam('Real spam 1', PATH_TO_REAL_SPAM_1_FILE, optimized, voc, voc_size)
        is_spam('Real spam 2', PATH_TO_REAL_SPAM_2_FILE, optimized, voc, voc_size)
\end{lstlisting}



\end{document}