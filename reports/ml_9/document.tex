\input{config/preamble}

\input{config/macros}

\begin{document}

\input{sections/title}

\section{Рекомендательные системы}
\label{sec:purpose}

Набор данных ex9_movies.mat представляет собой файл формата *.mat (т.е. сохраненного из Matlab). Набор содержит две матрицы Y и R - рейтинг 1682 фильмов среди 943 пользователей. Значение Rij может быть равно 0 или 1 в зависимости от того оценил ли пользователь j фильм i. Матрица Y содержит числа от 1 до 5 - оценки в баллах пользователей, выставленные фильмам.

\subsection{Загрузите данные ex9_movies.mat из файла.}

\begin{lstlisting}
        dataset = read_matlab_file(MOVIES_PATH)
        ratings = dataset.get("Y")
        user_rates = dataset.get("R")
\end{lstlisting}

\subsection{Выберите число признаков фильмов (n) для реализации алгоритма коллаборативной фильтрации.}

\begin{lstlisting}
        dataset = read_matlab_file(MOVIES_PATH)
        ratings = dataset.get("Y")
        user_rates = dataset.get("R")
\end{lstlisting}

\subsection{Реализуйте функцию стоимости для алгоритма.}

\begin{lstlisting}
        dataset = read_matlab_file(MOVIES_PATH)
        ratings = dataset.get("Y")
        user_rates = dataset.get("R")
\end{lstlisting}

\subsection{Реализуйте функцию вычисления градиентов.}

\begin{lstlisting}
        dataset = read_matlab_file(MOVIES_PATH)
        ratings = dataset.get("Y")
        user_rates = dataset.get("R")
\end{lstlisting}

\subsection{При реализации используйте векторизацию для ускорения процесса обучения.}

\begin{lstlisting}
        dataset = read_matlab_file(MOVIES_PATH)
        ratings = dataset.get("Y")
        user_rates = dataset.get("R")
\end{lstlisting}

\subsection{Добавьте L2-регуляризацию в модель.}

\begin{lstlisting}
        dataset = read_matlab_file(MOVIES_PATH)
        ratings = dataset.get("Y")
        user_rates = dataset.get("R")
\end{lstlisting}

\subsection{Обучите модель с помощью градиентного спуска или других методов оптимизации.}

\begin{lstlisting}
        dataset = read_matlab_file(MOVIES_PATH)
        ratings = dataset.get("Y")
        user_rates = dataset.get("R")
\end{lstlisting}

\subsection{Добавьте несколько оценок фильмов от себя. Файл movie_ids.txt содержит индексы каждого из фильмов.}

\begin{lstlisting}
        dataset = read_matlab_file(MOVIES_PATH)
        ratings = dataset.get("Y")
        user_rates = dataset.get("R")
\end{lstlisting}

\subsection{Сделайте рекомендации для себя. Совпали ли они с реальностью?}

\begin{lstlisting}
        dataset = read_matlab_file(MOVIES_PATH)
        ratings = dataset.get("Y")
        user_rates = dataset.get("R")
\end{lstlisting}

\end{document}