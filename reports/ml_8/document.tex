\input{config/preamble}

\input{config/macros}

\begin{document}

\input{sections/title}

\section{Выявление аномалий}
\label{sec:purpose}

Набор данных ex8data1.mat представляет собой файл формата *.mat (т.е. сохраненного из Matlab). Набор содержит две переменные X1 и X2 - задержка в мс и пропускная способность в мб/c серверов. Среди серверов необходимо выделить те, характеристики которых аномальные. Набор разделен на обучающую выборку (X), которая не содержит меток классов, а также валидационную (Xval, yval), на которой необходимо оценить качество алгоритма выявления аномалий. В метках классов 0 обозначает отсутствие аномалии, а 1, соответственно, ее наличие.

Набор данных ex8data2.mat представляет собой файл формата *.mat (т.е. сохраненного из Matlab). Набор содержит 11-мерную переменную X - координаты точек, среди которых необходимо выделить аномальные. Набор разделен на обучающую выборку (X), которая не содержит меток классов, а также валидационную (Xval, yval), на которой необходимо оценить качество алгоритма выявления аномалий.


\subsection{Загрузите данные ex8data1.mat из файла.}

\begin{lstlisting}
        dataset = read_matlab_file(DATA_PATH_1)
        x = dataset.get("X")
\end{lstlisting}

\subsection{Постройте график загруженных данных в виде диаграммы рассеяния.}

\begin{figure}[h]
\centering
    \includegraphics[totalheight=7cm]{1.png}
  \label{sec:purpose:payings}
\end{figure}

\subsection{Представьте данные в виде двух независимых нормально распределенных случайных величин.}

\begin{lstlisting}
def check_gauss(x):
    m, n = x.shape
    mu = (np.sum(x, 0) / m).reshape((n, 1))
    sigma_squared = (np.sum((x - mu.T) ** 2, 0) / m).reshape((n, 1))
    return mu, sigma_squared
\end{lstlisting}

\subsection{Оцените параметры распределений случайных величин.}

\begin{lstlisting}
def gaussian_distribution_density(x, mu, sigma_squared):
    k = mu.shape[0]

    if sigma_squared.shape[1] == 1 or sigma_squared.shape[0] == 1:
        sigma_squared = np.diag(sigma_squared.reshape(-1))

    x = x - mu.T
    return (2 * np.pi) ** (- k / 2) * np.linalg.det(sigma_squared) ** (-0.5) * \
           np.exp(-0.5 * np.sum(x @ np.linalg.pinv(sigma_squared) * x, 1)).reshape((x.shape[0], 1))
\end{lstlisting}

\subsection{Постройте график плотности распределения получившейся случайной величины в виде изолиний, совместив его с графиком из пункта 2.}

\begin{figure}[h]
\centering
    \includegraphics[totalheight=7cm]{3.png}
  \label{sec:purpose:payings}
\end{figure}

\begin{figure}[h]
\centering
    \includegraphics[totalheight=7cm]{4.png}
  \label{sec:purpose:payings}
\end{figure}

\subsection{Подберите значение порога для обнаружения аномалий на основе валидационной выборки. В качестве метрики используйте F1-меру.}

\begin{lstlisting}
def select_threshold(y, p):
    max_p = np.max(p)
    min_p = np.min(p)
    best_f_1 = None
    best_epsilon = None

    steps_count = 1000
    step_size = (max_p - min_p) / steps_count
    step_index = 0
    epsilon = min_p
    while step_index < steps_count:
        step_index = step_index + 1
        epsilon = epsilon + step_size
        cv_predictions = p < epsilon

        false_positives = np.sum((cv_predictions == 1) & (y == 0))
        true_positives = np.sum((cv_predictions == 1) & (y == 1))
        false_negatives = np.sum((cv_predictions == 0) & (y == 1))

        precision = true_positives / (true_positives + false_positives + 1e-10)
        recall = true_positives / (true_positives + false_negatives + 1e-10)
        f_1 = (2 * precision * recall) / (precision + recall + 1e-10)

        if best_f_1 is None or best_f_1 < f_1:
            best_f_1 = f_1
            best_epsilon = epsilon

    return best_epsilon, best_f_1
\end{lstlisting}

\subsection{Выделите аномальные наблюдения на графике из пункта 5 с учетом выбранного порогового значения.}

\begin{figure}[h]
\centering
    \includegraphics[totalheight=7cm]{2.png}
  \label{sec:purpose:payings}
\end{figure}

\subsection{Загрузите данные ex8data2.mat из файла.}

\begin{lstlisting}
        dataset = read_matlab_file(DATA_PATH_2)
        x = dataset.get("X")
        x_val = dataset.get("Xval")
        y_val = dataset.get("yval")
\end{lstlisting}

\subsection{Представьте данные в виде 11-мерной нормально распределенной случайной величины.}

\begin{lstlisting}
p = util.gaussian_distribution_density(x, mu, sigma_squared)
p_cv = util.gaussian_distribution_density(x_val, mu, sigma_squared)
\end{lstlisting}

\subsection{Подберите значение порога для обнаружения аномалий на основе валидационной выборки. В качестве метрики используйте F1-меру.}

\begin{lstlisting}
lab8                 - [INFO ] -- Generated epsilon = 1.377228890761358e-18 (F1 = 0.6153846153325445)
lab8                 - [INFO ] -- Generated epsilon = 1.377228890761358e-18 (F1 = 0.6153846153325445)
lab8                 - [INFO ] -- Generated epsilon = 1.377228890761358e-18 (F1 = 0.6153846153325445)
\end{lstlisting}


\end{document}