\section{Задача прогнозирования временных рядов}
\label{sec:purpose}

Прогнозирование временных рядов является одним из важных факторов предсказания будущих значений, анализе трендов, циклов и сезонностей в определённых
значениях. Для начала, следует рассмотреть само понятие временного ряда.

Временной ряд: \begin{equation}\label{timeline_value} Y_1, Y_2 ... Y_t \in \mathbf{R}  \end{equation}, значения признака, измеренные через постоянные временные интервалы.

Ключевая особенность состоит в том, что измерения признака происходит во времени и между разными измерениями всегда проходит одинаковое количество времени.
Т.к. если промежуток между отсчётами будет случайным, то в этом случае это будет являться случайным процессом и методы для обработки будут использоваться другие, нежели
при работе с прогнозированием временных рядов.

В данном случае, мы рассматриваем прогнозирование вещественного скалярного ряда, т.е. измерения принадлежат множеству \mathbf{R}.